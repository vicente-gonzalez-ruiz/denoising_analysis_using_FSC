\documentclass{article}
\usepackage{hyperref}
\usepackage{amsmath}

\title{Analysis of the performance of volume denoisers using FSC (Fourier Shell Correpation) curves}

\author{Vicente González Ruiz and José Jesús Fernández Rodríguez}

\begin{document}
\maketitle

\begin{abstract}
  The FSC curve is a measure of the similarity between two 3D volumes
  represented in the Fourier domain. Each point of the curve
  prepresents the correlation between two ``shells'' of Fourier
  coefficients of both volumes. An advantage of the FSC over other
  similarity metrics such as the
  \href{https://en.wikipedia.org/wiki/Mean_squared_error}{MSE (Mean
    Squared Error)}, the
  \href{https://en.wikipedia.org/wiki/Structural_similarity_index_measure}{SSIM
    (Structural Similarity Index Measure)} and the
  \href{https://en.wikipedia.org/wiki/Pearson_correlation_coefficient}{PPC
    (Pearson Correlation Coefficient)} is that FSC values depend on
  the frequency, and this can be interesting in some scenarios, such
  as structural biology. In this work we analyze the effect of some
  denoising algorithms applied to noisy electron microscopy volumes,
  and analyze this effect using the FSC curve, the MSE, and the SSIM.
\end{abstract}

\section{Definitions}
\begin{enumerate}
  \item 
Having two volumes $\mathbf{X}$ and $\mathbf{Y}$,
\begin{equation}
  \text{MSE}(\mathbf{X},\mathbf{Y}) = \frac{1}{N}\sum_{i=1}^N(\mathbf{X}_i - \mathbf{Y}_i)^2,
\end{equation}
where $N$ is the number of samples in one of the volumes (that must
have the same number of samples).

\item
After splitting the volume into $M$ non overlapped sub-vols,
\begin{equation}
  \text{SSIM}(\mathbf{X}, \mathbf{Y}) := \frac{1}{N} \sum_{i=1}^N \frac{(2\overline{\mathbf{x}}_i \overline{\mathbf{y}}_i + c_1)(2\sigma_{\mathbf{x}_i \mathbf{y}_i} + c_2)}{(\overline{\mathbf{x}_i^2} + \overline{\mathbf{y}_i^2} + c_1)(\sigma^2_{\mathbf{x}_i} + \sigma^2_{\mathbf{y}_i} + c_2)},
\end{equation}
where $\overline{x}i$ is the mean of the $i$-th sub-vol of
$\mathbf{X}$, $\sigma^2_{\mathbf{x}_i}$ is its variance (equivalently
for $\mathbf{Y}$), $\sigma_{\mathbf{x}_i\mathbf{y}_i}$ is the
covariance of both sub-vols, $c_1=(k_1L) ²$, $c_2=(k_2L) 2$ are two
variables to stabilize the division with weak denominator, $L$ is the
dynamic range of the voxel values, with constants with default vales
$K_1=0.01$ , and $k_2=0.03$, and where the default size of the local
sub-vols is $7\times 7\times 7$.

\item
The PPC is given by
\begin{equation}
  \text{PPC}(\mathbf{X}, \mathbf{Y}) := \frac{\sum_i(\mathbf{X}_i - \overline{\mathbf{X}})(\mathbf{Y}_i - \overline{\mathbf{Y}})}{\sqrt{\sum_i (\mathbf{X}_i - \overline{\mathbf{X}})^2 \sum_i (\mathbf{Y}_i - \overline{\mathbf{Y}})^2}}
\end{equation}

\item
A FSC value of the FSC curve is determined by~\cite{verbeke2024self}
\begin{equation}
\text{FSC}(\mathbf{X}, \mathbf{Y}; r) := \frac{\sum_{i \in S_r} (\mathcal{F}(\mathbf{X)}_i \mathcal{F}(\mathbf{Y)}_i^*)}{\sqrt{\sum_{i \in S_r} |\mathcal{F}(\mathbf{X})_i|^2 \sum_{i \in S_r} |\mathcal{F}(\mathbf{Y})_i|^2}},
\end{equation}
where $i=(x, y, z)$ is a point (a coefficient in the Fourier domain)
of the surface of the sphere $S_r$ defined by $x^2+y^2+z^2=r^2$,
$\mathcal{F}(\cdot)$ is the Fourier transform of $\cdot$, and
$\cdot^*$ denotes the complex conjugate of $\cdot$.
\end{enumerate}

\section{Noise model}

Being $\mathbf{X}$ the original (without noise, ground truth usually unknown) volume,
\begin{equation}
  \mathbf{Y} = \mathbf{X} + \mathbf{N}
  \label{eq:noisy_model}
\end{equation}
is the noisy (known) volume, where $\mathbf{N}$ is the added (unknown)
noise. We assume that $\mathbf{X}$ and $\mathbf{N}$ are statistically
independent. The idea is compute
$\text{metric}(\mathbf{Y}, \hat{\mathbf{Y}})$ because $\mathbf{N}$ can
only be estimated and therefore, it is impossible to obtain
$\mathbf{X}$ from Eq.~\ref{eq:noisy_model}.

\section{GD (Gaussian Denoising)}

Gaussian denoising works weight-averaging signal samples. Under the
assumption that most of the energy (and information) is concentrated
in the low frequencies, a Gaussian filter is defined by
\begin{equation}
  \mathbf{h} = \{\mathbf{h}_i\} = \frac{1}{\sqrt{2\pi}\sigma}e^{{-i}^2/(2\sigma^2)},
\end{equation}
where $\sigma$ is the standard deviation of a normal distribution. The
filtering is represented by
\begin{equation}
  \hat{\mathbf{Y}} = \mathbf{Y}*\mathbf{h},
  \label{eq:GF}
\end{equation}
where $\cdot*\cdot$ represent the convolution of digital signals.

Multidimensional Gaussian filtering is separable, which means that we
can apply the 1D filter to all the dimensions of the signal.

GD requires $\sigma$ to be provided as a parameter. A higher $\sigma$
results in a lower the cut-off frequency of the filter. $\sigma$ is
not (objectively) optimizable.

% \section{AND}

% \section{N2V}

\section{Cryo-CARE \cite{buchholz2019cryo}}

\section{BM4D}

\section{SDPG (Structure-Preserving Gaussian Denoising) \cite{gonzalez2023structure}}

In SDPG the Eq.~\ref{eq:GF} holds, but using an 2D OF (Optical Flow) estimator
the signal samples are extracted from warped images generated
after projecting the original images with the vectors fields produced
the estimator. This decreases the bluring in those areas where the
estimator is able to detect structures.

The estimator requires the following parameters to be defined:
\begin{enumerate}
\item $\sigma$, with an impact similar to the case of GD.
\item $w$, the size of a 2D window used to analyze the 2D slices. If
  $w$ is large, the estimator is less sensitive to the noise but also
  the small structures are ignored.
\item $l$, that controls the maximun displacements for the
  estimator. Higher $l$ values increase blurring.
\end{enumerate}

\section{2D-RSVD (2D Random Shuffing Volume Denoising}

\section{3D-RSVD (3D Random Shuffling Volume Denoising}


\section{}

\bibliographystyle{plain}
\bibliography{signal_processing}

\end{document}